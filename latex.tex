\documentclass[12pt]{article}
\usepackage{natbib,hyperref,fullpage,times,url}
\usepackage{amssymb,amsmath,mathptmx,mathrsfs}
\usepackage{graphicx, epstopdf}
\DeclareGraphicsExtensions{.eps}
\title{Title 3}
\author{Rugal Bernstein\\ \href{mailto:yaoy@cs}{yaoy345@ca}}
\date{\today}
\bibliographystyle{plain}
\begin{document}

\maketitle
\thispagestyle{empty}

\section{Max Margin Solution (20 points)}
\subsection*{(a) Using your knowledge of the maximum margin boundary, write down a vector that points in the same direction as $w$}
Answer:\\\\
$$
\begin{aligned}
\phi(x_1) &= [1,0,0]\\
\phi(x_2) &= [1,2,2]
\end{aligned}
$$
\includegraphics[scale=0.9]{w_vector}\\ 
The plot above is line between $\phi(x_1)$ and $\phi(x_2)$. Since the hyperplane will separate these two points with the margin distance maximized, this line is actually the normal of this hyperplane. so I can say vector $$v=\phi(x_2) - \phi(x_1)=[0,2,2]$$ will point to the same direction as $w$.\\ 

\subsection*{(b) Derive the actual solution $w$ and $b$	}
Answer:\\\\
The hyperplane goes through point $(1,1,1)$ and its normal is $[0,1,1]$, hence the decision boundary as below:\\
$$
[0,1,1] \cdot [x-1, y-1,z-1]=0\\
$$
$$
y+z-2=0
$$
Hence, as shown above, $w=[0,1,1]$, and bias term $b=-2$.
\subsection*{(c) What is the value of the margin?}
Answer:\\\\
$$
2\gamma = ||v|| = 2\sqrt{2}
$$
\subsection*{(d) Write down the decision boundary $f(x)$ as a function of $x$.}
Answer :\\\\
\includegraphics[scale=0.8]{decision_boundary}\\ 
$$
\begin{aligned}
f(x)&=\sqrt{2}x+x^2-2\\
&=\left(x+\frac{\sqrt{2}}{2}\right)^2-\frac{5}{2}
\end{aligned}
$$
For large negative $x$ value, it will be classified as $y=1$.\\\\

The discriminative function does not equal zero at the midpoint because this question employed kernel, those original data were mapped onto another dimension.
\section{Kernels (10 points)}
\subsection*{(a) Is this function a kernel?}
Firstly I need to split the whole document into words, use every single word lowercased as a element and form the whole words sequence as a vector, followed by trim the comma and full stop.\\

"This is a sentence, a good sentence." $\rightarrow$ [this, is, a, sentence, a, good, sentence]\\\\
Then I need to sort this vector by alphabetic order, and eliminate the duplicate element.\\

[This, is, a, sentence, a, good, sentence] $\rightarrow$ [a, good, is, sentence, this]\\\\
AFter that, I need to hash elements of this vector into a dictionary table, if in that the element appears particular position, mark 1, otherwise, mark 0.\\

[a, good, is, sentence, this] $\rightarrow$mapped by dictionary$\rightarrow$ [...1, ...1, ...0, 1, ...1, 0, ...1...]\\\\
Althought this will be a very big vector after hash check with dictionary. Now this is our function $\phi(x)$.\\
When I use $\phi(x)\phi(z)$, it will actually compute the number of unique words that occur in both $x$ and $z$. The return value is the number of word.\\\\
So I think this is a kernel function.

\subsection*{(b) Show that the following is a kernel}
$$
\begin{aligned}
k_{\beta}(\textbf{x},\textbf{z})&=(1+\beta \textbf{x}^T \textbf{z})^2-1\\
&=(\beta \textbf{x}^T \textbf{z})^2+2\beta \textbf{x}^T \textbf{z}\\
&=\left(
\begin{array}{c}
\beta\textbf{x}^2\\
\sqrt{2\beta}\textbf{x}
\end{array}
\right) \cdot
\left(
\begin{array}{c}
\beta\textbf{z}^2\\
\sqrt{2\beta}\textbf{z}
\end{array}
\right)
\end{aligned}
$$
So I think the feature map is 
$$
\Phi(\textbf{x})=
\left(
\begin{array}{c}
\beta\textbf{x}^2\\
\sqrt{2\beta}\textbf{x}
\end{array}
\right)
$$
Hence this $k_{\beta}(\textbf{x},\textbf{z})$ is kernel function.



\end{document}
